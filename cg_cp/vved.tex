\chapter*{Введение}
\addcontentsline{toc}{chapter}{Введение}

Компьютерную графику как науку можно назвать одной из самых распространенных и обширных сфер цифрового мира. Подавляющее число областей жизни человека, связанных с компьютером, так или иначе пересекаются с компьютерной графикой. Одна из самых очевидных направлений для применения компьютерной графики – фотография.

Получение изображений близких по своим характеристикам к «живым» картинам – одна из сложнейших, но, в то же время, одна из наиболее актуальных задач компьютерной графики. Одной из причин сложности генерации реалистичных изображений является необходимость учитывать сложные принципы распространения света.

Целью данной работы является выбрать алгоритмы для эффективной визуализации студийного освещения, а также обосновать этот выбор. Реализовать данные алгоритмы, а также, по возможности, адаптировать их к условиям задачи.  