\chapter{Технологическая часть}

В данной главе был выбран язык программирования, описана структура классов программы и разработана ее диаграмма, был описан интерфейс готовой программы и приведены примеры работы программы.

\section{Выбор языка программирования}

Для реализации данной программы был выбран язык программирования  C/C++ \cite{Cpp}. 

Следующие факторы повлияли на этот выбор.

\begin{enumerate}
	\item Выбранный язык программирования предоставляет широкие возможности для управления ресурсами (в первую очередь выделяемой памятью).
	\item Данный язык является строго типизированным. Этот факт облегчает тестирование.
	\item Программа на языке программирования C/C++ не обязана иметь фиксированную структуру (весь код, теоретически, может быть помещен в один файл).
\end{enumerate}

\section{Структура классов программы}

При реализации данного программного обеспечения было выделено несколько математических абстракций, которые были описаны пользовательскими классами:

\begin{itemize}
	\item $Object$ – абстракция базового объекта; 
	\item $Vertex$ – абстракция вершины;
	\item $Vertexes$ – абстракция множества связанных вершин;
	\item $Triangle$ – формализация связи трех вершин;
	\item $Triangles$ – множество связанных формализаций;
	\item $Figure$ – множество связанных вершин и формализаций, образующее обособленную «фигуру»;
	\item $LightSource$ – абстракция источника света;
	\item $Camera$ – абстракция камеры;
	\item $Scene$ – множество обособленных «фигур».
\end{itemize}

\section{Диаграмма классов}

Диаграмма классов данного проекта представлена на рисунке  \ref{img:classes}.

\img{100mm}{classes}{Диаграмма классов}

\section{Интерфейс программного обеспечения}

Основные возможности предоставляемые пользователю со стороны интерфейса:

\begin{itemize}
	\item добавление графического примитива (куб, шар);
	\item добавление источника света;
	\item поворот сцены;
	\item смещение сцены;
	\item масштабирование сцены.
\end{itemize}

Изображение интерфейса программы представлено на рисунке \ref{img:interface}.

\img{80mm}{interface}{Интерфейс программы}

\newpage
\section{Результаты работы программного обеспечения}

В данном разделе представлены изображения результатов работы программы (отдельных ее частей и всего интерфейса целиком).

На рисунке \ref{img:lowpoli} представлено изображение шара, сгенерированного с малой степенью детализации. 
\img{80mm}{lowpoli}{Малая степень детализации шара}

\newpage
На рисунке \ref{img:midpoli} представлено изображение шара, сгенерированного с средней степенью детализации. 
\img{80mm}{midpoli}{Средняя детализация шара}

На рисунке \ref{img:highpoli} представлено изображение шара, сгенерированного с высокой степенью детализации. 
\img{90mm}{highpoli}{Высока степень детализации шара}

Из представленных изображений видно, что с повышением детализации модели, недостатки модели освещения Ламберта становятся менее заметными.

На рисунке \ref{img:running} представлен интерфейс программы во время выполнения. В данном примере был создан один источник света, четыре шара разных цветов. 
\img{90mm}{running}{Интерфейс программы во время выполнения}

\section*{Вывод}

В этом разделе был выбран язык программирования, рассмотрена uml-диаграмма основных классов, подробно разобран интерфейс приложения и приведены результаты работы програмы.

