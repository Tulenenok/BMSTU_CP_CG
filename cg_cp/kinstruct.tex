\chapter{Конструкторская часть}

В данной главе были детально рассмотрены все этапы работы программы, а также разработаны схемы некоторых алгоритмов, которые используются в программе.

\section{Общий алгоритм решения поставленной задачи}

Общий алгоритм решения поставленной задачи выглядит следующим образом:
\begin{itemize}
	\item задать объекты сцены (фоновые плоскости, источники света, тестируемый объект);
	\item рассчитать освещенность поверхностей сцены;
	\item изобразить сцену.
\end{itemize}

\section{Алгоритм использующий Z-буфер}

Для визуализации сцены был выбран алгоритм построчного сканирования с Z-буфером.

Концепция Z-буфера является обобщением идеи буфера кадра, который используется для запоминания атрибутов пикселей в пространстве кадра, в то время как Z-буфер хранит единственный атрибут – глубину пиксела относительно картинной плоскости. (Картинная плоскость, исторически, помещается в плоскость XOY, поэтому перпендикулярное этой плоскости направление, как глубина, легло в название буфера.) 

Изначально Z-буфер инициализируется максимальным значение глубины для данной системы объектов или «бесконечностью». Задача алгоритма, далее, сводится к банальному поиску минимального значения глубины для каждого пиксела и отображению его на экран с учетом характеристик объекта, к которому он принадлежит.

На рисунке \ref{img:z-buffer-classic} представлена схема алгоритма использующего Z-буфер.

\img{200mm}{z-buffer-classic}{Схема алгоритма удаления невидимых граней и поверхностей, использующего Z-буфер (последовательная версия)}

В данном случае, количество граней полигона строго фиксируется значением три. «Преимущества данного представления: простое разбиение массива точек на треугольники, обеспечение хорошей реалистичности отображения поверхности воды вследствие возможности детального моделирования рельефной поверхности, удобство в использовании треугольников в алгоритме построчного сканирования, использующего Z-буфер, из-за сведения оперирования многоугольниками к оперированию треугольниками.» --- К.А. Якиль.

Недостатки алгоритма --- большой объем требуемой для работы памяти. 

При работе с данным алгоритмом приходится столкнуться с трудоемкостью реализации эффектов, связанных с полупрозрачностью, и ряда других специальных задач, повышающих реалистичность изображения. Поскольку алгоритм заносит пиксели в буфер кадра в произвольном порядке, то довольно сложно получить информацию, которая необходима для методов, основывающихся на предварительном анализе сцены.

Преимущества алгоритма --- алгоритм достаточно «простой» для понимания и реализации и алгоритм решает задачу не только удаления невидимых поверхностей, но и визуализации пересечения поверхностей.

Большой объем требуемой памяти этого алгоритма частично исправляется тем фактом, что он будет использоваться в построчном сканировании, а значит объем выделяемой памяти будет соизмерим с шириной экрана, а не с его площадью, как это описано в классической реализации.

Реализация эффектов полупрозрачности не ставится основным требованием для данной работы, а лишь служит возможным ее улучшением, что дает нам возможность не рассматривать эту проблему.

\section{Основной цикл программы}

После выполнения просчета освещенности поверхностей наступает этап последовательной отрисовки сцены под разными углами, которые задаются пользователем.

Общий алгоритм работы программы:
\begin{itemize}
	\item рассчитать необходимые углы поворота сцены;
	\item для каждого угла, отрисовать сцену, повернутую на значение этого угла;
	\item для каждого угла, сохранить получившееся изображение.
\end{itemize}

\section{Модель освещения Ламберта}

Данная модель вычисляет цвет поверхности в зависимости от того как на нее светит источник света. Согласно данной модели, освещенность точки равна произведению силы источника света и косинуса угла, под которым он светит на точку.

\section{Генерация тестируемой модели}

Тестируемую модель задает пользователь. Тестируемая модель состоит из источников освещения и графических примитивов (кубов и шаров).

\section*{Вывод}
В данной главе были детально рассмотрены все этапы работы программы, а также разработаны схемы некоторых алгоритмов, которые используются в программе.

