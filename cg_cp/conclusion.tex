\chapter*{Заключение}
\addcontentsline{toc}{chapter}{Заключение}

В рамках данного курсового проекта были: 
\begin{itemize}
	\item описаны структуры трехмерной сцены, включая объекты, из которых состоит сцена;
	\item проанализированы и выбраны необходимые существующие алгоритмы для построения сцены;
	\item проанализированы и выбраны варианты оптимизации ранее выбранного алгоритма удаления невидимых линий; 
	\item реализованы выбранные алгоритмы; 
	\item разработано программное обеспечение, которое позволит отобразить трехмерную сцену; 
	\item проведены сравнение стандартного и реализованного оптимизированного алгоритма удаления невидимых линий. 
\end{itemize}

В ходы выполнения эксперимента было показано, что многопоточная реализация алгоритма удаления невидимых граней, использующего Z-буфер, может показывать меньшее время выполнения, чем однопоточная или линейная реализация. Также было проведено сравнение времени выполнения программ реализующих данный алгоритм на языках программирования C/C++ и Python.