\chapter{Аналитическая часть}

В данной главе были рассмотрены объекты сцены, формы и способы задания трехмерных моделей, алгоритмы удаления невидимых ребер и поверхностей и модели освещения. На основании проведенного анализа были выбраны форма и способ задания трехмерных моделей, алгоритм удаления невидимых ребер и поверхностей и модель освещения, которые будут использоваться при реализации программы. 

\section{Описание объектов сцены}

В данном разделе были описаны сущности, которые будут отображаться в процессе выполнения программы.


Сцена включает в себя:
\begin{itemize}
	\item один или несколько источников света;
	\item одну или несколько фоновых плоскостей;
	\item тестируемую модель.
\end{itemize}


Источник света представляется в виде материальной точки, которая испускает лучи во все стороны. Если он устанавливается на «бесконечном» удалении от сцены, на сцену попадает множество параллельных лучей. В программе источники света будут играть роль студийных проекторов, однако визуализировать их не планируется.

Фоновые плоскости – условные границы сцены, которые также будут отображаться на итоговом изображении.

Тестируемая модель – некоторая объемная модель, которая будет использоваться для предпросмотра эффективности студийного освещения. В качестве тестируемой модели могут выступать как примитивы (кубы или пирамиды), так и более сложные модели (каркасы людей или животных).

\section{Анализ и выбор формы задания трехмерных моделей}

В данном разделе были рассмотрены формы задания трехмерных моделей. На основании проведенного анализа была выбрана одна из форм, которая будет использоваться при реализации программы.

В данной работе под моделью понимается отображение некоторых физических свойств объектов реального мира (размера и формы) в цифровом пространстве.

Выделяют три основных вида моделей:
\begin{itemize}
	\item каркасная модель;
	\item поверхностная модель;
	\item объемная модель.
\end{itemize}

Рассмотрим основные характеристики этих моделей \cite{model}.

\subsection{Каркасная модель}

Каркасная модель --- простейшая из представленных моделей, объекты
задаются множеством точек в пространстве и связями между этими точками.

Основным преимуществом каркасной модели можно назвать простоту, к недостатком отнесем то, что такие модели порой бывает невозможно отобразить без искажения восприятия форм и размеров.

\subsection{Поверхностная модель}

Поверхностная модель \cite{surface} --- как следует из названия, в этой модели объекты задаются в виде набора плоскостей. 

Эта модель позволяет с достаточной точностью передать физические характеристики объекта, однако, если объект становится слишком сложным, могут возникнуть трудности с определением расположения материала внутри модели.

\subsection{Объемная модель}

Объемная модель --- схожа с поверхностной моделью, однако добавляется список внутренних нормалей, благодаря которому мы можем однозначно понять где находится материал модели.

Такая модель хранит исчерпывающую информацию об объекте, что хорошо при работе со сложными сценами, однако зачастую излишне.

\subsection*{Вывод}

Подводя итоги, можно сделать вывод, что оптимальным решением будет использование поверхностной модели, так как сложность предполагаемой сцены не требует хранения дополнительной информации об ориентированности плоскостей.


\section{Анализ способа задания поверхностных моделей}

В данном разделе были рассмотрены способы задания поверхностных моделей. На основании проведенного анализа была выбрана одна из них, который будет использоваться при реализации программы.

Существует два распространённых способа задания поверхностных моделей: аналитический и полигональный.

\subsection{Аналитический способ}

При аналитическом способе хранения поверхностных моделей возникают дополнительные ресурсные траты на получение фактических значений вершин полигонов модели;

\subsection{Полигональный способ}

При полигональном способе хранения не возникают трудности при получении фактических координат вершин полигонов модели, что может облегчить написание алгоритмов.

Возможные способы хранения информации о сетке:

\begin{enumerate}
	\item Список граней.
	
	Объект – это множество граней и вершин. В каждую грань входят как минимум 3 вершины.
	
	\item «Крылатое» представление.
	
	Каждая точка ребра указывает на две вершины, две грани и четыре ребра, которые её касаются.
	
	\item Полурёберные сетки.
	
	То же «крылатое» представление, но информация обхода хранится для половины грани.
	
	\item Таблица углов.
	
	Таблица, хранящая вершины. Обход заданной таблицы неявно задаёт полигоны. Такое представление более компактно и более производительно для нахождения полигонов, но, в связи с тем, что вершины присутствуют в описании нескольких углов, операции по их изменению медленны.
	
	\item Вершинное представление.
	
	Хранятся лишь вершины, которые указывают на другие вершины. Простота представления даёт возможность проводить над сеткой множество операций.
\end{enumerate}

\subsection*{Вывод}

Было принято решение при реализации итогового продукта использовать полигональную сетку в виде списка полигонов из трех вершин. Такой подход позволит эффективно описывать сложные модели, при этом преобразования будут применяться достаточно просто и быстро. Так же, явное описание вершин модели упрощает написание алгоритма удаления невидимых граней.

\section{Анализ и выбор алгоритма удаления невидимых ребер и поверхностей}

В данном разделе были рассмотрены алгоритмы удаления невидимых ребер и поверхностей. На основании проведенного анализа был выбран один из них, который будет использоваться при реализации программы.

Выдвигаемые требования к свойствам алгоритма для обеспечения реалистичного изображения.

\begin{enumerate}
	\item Алгоритм может работать как в объектном пространстве, так и в пространстве изображений.
	\item Алгоритм должен быть достаточно быстрым.
	\item Алгоритм должен иметь высокую реалистичность изображения.
\end{enumerate}

\subsection{Алгоритм, использующий Z-буфер}

Суть данного алгоритма – это использование двух буферов: буфера кадра, в котором хранятся атрибуты каждого пикселя, и Z-буфера, в котором хранятся информация о координате Z для каждого пикселя \cite{zbuffer}.

Первоначально в Z-буфере находятся минимально возможные значения Z, а в буфере кадра располагаются пиксели, описывающие фон. Каждый многоугольник преобразуется в растровую форму и записывается в буфер кадра.

В процессе подсчета глубины нового пикселя, он сравнивается с тем значением, которое уже лежит в Z-буфере. Если новый пиксель расположен ближе к наблюдателю, чем предыдущий, то он заносится в буфер кадра и происходит корректировка Z-буфера.

Преимущества алгоритма --- простота реализации и оценка трудоемкости линейна.
Недостатки алгоритма --- сложная реализация прозрачности и большой объем требуемой памяти.

Данный алгоритм отвечает заявленным требованиям. Сложная реализация прозрачности, в данном случае, не существенна, так как изначально прозрачные объекты не планировались к рассмотрению.

\subsection{Алгоритм обратной трассировки лучей}

Суть данного алгоритма состоит в том, что наблюдатель видит объект с помощью испускаемого света, который согласно законам оптики доходит до наблюдателя некоторым путем. Отслеживать пути лучей от источника к наблюдателю неэффективно с точки зрения вычислений, поэтому наилучшим способом будет отслеживание путей в обратном направлении, то есть от наблюдателя к объекту.

Преимущества алгоритма:
\begin{itemize}
	\item высокая реалистичность синтезируемого изображения;
	\item работа с поверхностями в математической форме;
	\item вычислительная сложность слабо зависит от сложности сцены.
\end{itemize}

Недостатки алгоритма --- алгоритм имеет низкую производительность.

\subsection{Алгоритм Робертса}

Данный алгоритм работает в объектном пространстве, решая задачу только с выпуклыми телами.

Алгоритм выполняется в 3 этапа.

\begin{enumerate}
	\item Этап подготовки исходных данных.
	
	На данном этапе должна быть задана информация о телах. Для каждого тела сцены должна быть сформирована матрица тела $V$. Размерность матрицы - $4 * n$, где $n$ – количество граней тела.
	
	Каждый столбец матрицы представляет собой четыре коэффициента уравнения плоскости $ax + by + cz + d = 0$, проходящей через очередную грань.
	
	Матрица тела должна быть сформирована корректно, то есть любая точка, расположенная внутри тела, должна располагаться по положительную сторону от каждой грани тела. В случае, если для очередной грани условие не выполняется, соответствующий столбец матрицы надо умножить на $-1$. 
	
	\item Этап удаления рёбер, экранируемых самим телом.
	
	На данном этапе рассматривается вектор взгляда $E = {0, 0, -1, 0}$.
	Для определения невидимых граней достаточно умножить вектор $E$ на матрицу тела $V$. Отрицательные компоненты полученного вектора будут соответствовать невидимым граням.
	
	\item Этап удаления невидимых рёбер, экранируемых другими телами сцены.
	
	На данном этапе для определения невидимых точек ребра требуется построить луч, соединяющий точку наблюдения с точкой на ребре. Точка будет невидимой, если луч на своём пути встречает в качестве преграды рассматриваемое тело.
	
\end{enumerate}

Преимущества алгоритма --- работа в объектном пространстве и высокая точность вычисления.

Недостатки алгоритма:
\begin{itemize}
	\item рост сложности алгоритма --- квадрат числа объектов;
	\item тела сцены должны быть выпуклыми (усложнение алгоритма, так как нужна будет проверка на выпуклость);
	\item сложность реализации.
\end{itemize}

Алгоритм Робертса не подходит для решения поставленной задачи из-за высокой сложности реализации как самого алгоритма, так и его модификаций.

\subsection{Алгоритм художника}

Данный алгоритм работает аналогично тому, как художник рисует картину --- то есть сначала рисуются дальние объекты, а затем более близкие. Наиболее распространенная реализация алгоритма --- сортировка по глубине, которая заключается в том, что произвольное множество граней сортируется по ближнему расстоянию от наблюдателя, а затем отсортированные грани выводятся на экран в порядке от самой дальней до самой ближней. Данный метод работает лучше для построения сцен, в которых отсутствуют пересекающиеся грани.

Преимущество алгоритма --- требуем меньше памяти, чем, например, алгоритм Z-буффера.

Недостатки алгоритма --- недостаточно высока реалистичность изображения и сложность реализации при пересечения граней на сцене.

Данный алгоритм не отвечает главному требованию – реалистичность изображения. Также алгоритм художника отрисовывает все грани (в том числе и невидимые), на что тратится большая часть времени.

\subsection{Алгоритм Варнока}

Алгоритм Варнока является одним из примеров алгоритма, основанного на разбиении картинной плоскости на части, для каждой из которых исходная задача может быть решена достаточно просто.

Поскольку алгоритм Варнока нацелен на обработку картинки, он работает в пространстве изображения. В пространстве изображения рассматривается окно и решается вопрос о том, пусто ли оно, или его содержимое достаточно просто для визуализации. Если это не так, то окно разбивается на фрагменты до тех пор, пока содержимое фрагмента не станет достаточно простым для визуализации или его размер не достигнет требуемого предела разрешения.

Сравнивая область с проекциями всех граней, можно выделить случаи, когда изображение, получающееся в рассматриваемой области, определяется сразу:

\begin{itemize}
	\item проекция ни одной грани не попадает в область;
	\item проекция только одной грани содержится в области или пересекает область. В этом случае проекции грани разбивают всю область на две части, одна из которых соответствует этой проекции;
	\item существует грань, проекция которой полностью накрывает данную область, и эта грань расположена к картинной плоскости ближе, чем все остальные грани, проекции которых пересекают данную область. В данном случае область соответствует этой грани.
\end{itemize}

Если ни один из рассмотренных трех случаев не имеет места, то снова разбиваем область на четыре равные части и проверяем выполнение этих условий для каждой из частей. Те части, для которых таким образом не удалось установить видимость, разбиваем снова и т.д.

Преимущество алгоритма --- меньшие затраты по времени в случае области, содержащий мало информации.

Недостатки алгоритма --- алгоритм работает только в пространстве изображений и большие затраты по времени в случае области с высоким информационным содержимым.

Данный алгоритм не отвечает требованию работы как в объектном пространстве, так и в пространстве изображений, а также возможны большие затраты по времени работы.

\subsection*{Вывод}

Для решения поставленной задачи был выбран алгоритм использующий Z-буфер. Данный алгоритм позволит получить приемлемую реалистичность при достаточно простой реализации. В данном проекте, единственным недостатком этого алгоритма можно назвать большие затраты памяти, однако эта проблема не столь существенна на современных компьютерах.

\section{Анализ и выбор модели освещения}

В данном разделе были рассмотрены модели освещения. На основании проведенного анализа была выбрана одина из них, который будет использоваться при реализации программы.

Рассмотрим две модели освещения, а именно: модель Ламберта и модель Фонга.

\subsection{Модель Ламберта}

Модель Ламберта моделирует идеальное диффузное освещение, то есть свет при попадании на поверхность рассеивается равномерно во все стороны. При такой модели освещения учитывается только ориентация поверхности (N) и направление источника света (L) (Рисунок \ref{img:lambert}). 

\newpage
\img{60mm}{lambert}{Направленность источника света}

Эта модель является одной из самых простых моделей освещения и очень часто используется в комбинации с другими моделями. Она может быть очень удобна для анализа свойств других моделей, за счет того, что ее легко выделить из любой модели и анализировать оставшиеся составляющие.

\subsection{Модель Фонга}

Это классическая модель освещения. Модель представляет собой комбинацию диффузной и зеркальной составляющих. Работает модель таким образом, что кроме равномерного освещения на материале могут появляться блики. Местонахождение блика на объекте определяется из закона равенства углов падения и отражения. Чем ближе наблюдатель к углам отражения, тем выше яркость соответствующей точки.

\newpage
\img{60mm}{fong}{Направление источника света, отраженного луча и наблюдателя}

Падающий и отраженный лучи лежат в одной плоскости с нормалью к отражающей поверхности в точке падения (Рисунок \ref{img:fong}). Нормаль делит угол между лучами на две равные части. L – направление источника света, R – направление отраженного луча, V – направление на наблюдателя.

\subsection*{Вывод}

Модель Ламберта выглядит более эффективной с точки зрения производительности. В рамках поставленной задачи она является достаточной. Огромным плюсом также служит тот факт, что работа программы будет несколько быстрее, чем при использовании модели Фонга. Таким образом было принято решение использовать модель Ламберта.

\section*{Вывод}

Для реализации итогового программного продукта было принято решение использовать поверхностную модель для описания объектов сцены, для хранения поверхностей использовать полигональную сетку (каждый полигон --- три узла). В качестве алгоритма удаления невидимых граней и поверхностей был выбран алгоритм, использующий Z-буфер. В качестве модели освещения была выбрана модель Ламберта. 




