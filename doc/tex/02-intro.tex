\chapter*{Введение}
\addcontentsline{toc}{chapter}{Введение}

В современном мире компьютерная графика все больше проникает в обычную человеческую жизнь и используется в различных сферах. Типичные области ее применения – кинематография, компьютерные игры, наглядное отображение различных данных, а также моделирование экспериментов.

Одним из быстро развивающихся направлений компьютерной графики является моделирование и визуализация реалистичного трехмерного изображения. Для удовлетворения растущих потребностей в скорости синтеза и реалистичности полученного изображения разрабатываются новые алгоритмы и совершенствуются уже существующие.

Целью данной курсовой работы является обоснование выбора алгоритмов, которые можно использовать для получения трехмерных моделей растений с помощью фракталов, их практическая реализация и адаптация (при необходимости) к условиям решаемой задачи.
