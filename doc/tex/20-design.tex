\chapter{Конструкторский раздел}
В данном разделе были рассмотрены требования к программе и алгоритмы визуализации модели растения.

\section{Общий алгоритм решения поставленной задачи}
Алгоритм решения поставленной задачи выглядит следующим образом:
\begin{enumerate}
	\item Задать набор правил для построения модели растения.
	\item Рассчитать параметры поверхностей отдельных частей растения согласно установленным правилам.
	\item Соединить полученные поверхности в единую модель.
        \item Изобразить модель растения на сцене.
\end{enumerate}


\section{Разработка алгоритма Z-буффера}

Формальное описание алгоритма:
\begin{enumerate}
    \item Заполнить буфер кадра фоновым значением интенсивности.
    \item Заполнить z-буфер минимальным значением z.
    \item Для каждого пиксела $(x, y)$ грани вычислить его глубину $z(x, y)$ и
сравнить вычисленное значение со значением $z-$буфер$(x, y)$,
хранящимся в z-буфере в этой же позиции.
    \item Если $z(x, y) > z-буфер(x, y)$, то записать атрибут этого многоугольника (интенсивность, цвет и т.п.) в буфер кадра и заменить $z-$буфер$(x, y)$ на $z(x, y)$.
\end{enumerate}

Схема алгоритма представлена на рисунке \ref{img:z_buffer} соответственно.

\img{240mm}{z_buffer}{Схема алгоритма Z-буфера}

% \img{90mm}{z_buffer}{Пример работы алгоритма Z-буфера}


\section{Разработка алгоритма освещения Ламберта}
Данная модель вычисляет цвет поверхности в зависимости от того как на нее светит источник света. Согласно данной модели, освещенность точки равна произведению силы источника света и косинуса угла, под которым он светит на
точку.

Схема алгоритма простой закраски, основанного на принципах
модели освещения Ламберта, представлена на рисунке \ref{img:lamb}.

\img{165mm}{lamb}{Схема алгоритма простой закраски}

\section{Разработка алгоритма генерации растения}

Для генерации моделей растений будем использовать L-системы.

Lindenmayer System или просто L-system \cite{l-systems} – это математическая модель, предложенная в 1968 году венгерским биологом и ботаником Аристидом
Линденмайером, для изучения развития простых многоклеточных организмов.

Позже эта модель была расширена и стала использоваться для моделирования сложных ветвящихся структур – разнообразных деревьев и цветов.

Основная идея L-систем – постоянная перезапись элементов строки (rewriting). Другими словами, это способ получения сложных объектов путем замены частей простого начального объекта по некоторым правилам. 

Подобный подход удобен для описания растений, так как они обладают свойствами рекурсивности и самоподобия при росте. Например, маленькие листочки, являющиеся частью большого взрослого составного листа, имеют ту же форму, что весь лист имел на раннем этапе формирования.

Язык L-систем очень прост, он состоит из символов (алфавита) и
продукционных правил. Первое состояние предложения называется аксиомой.
К этой аксиоме можно применить продукционные правила для эволюции или
роста системы. Например, если у нас есть система с аксиомой $A$ и единственным
правилом $A$ -> $ABB$, то после первой итерации предложение сменится на $ABB$,
потом на $ABBBB$ и т.д.

\subsection{Классификация L-систем}
% L-системы имеют свою классификацию от простых до сложных:
\begin{enumerate}
    \item Детерминированные контекстно-свободные L-системы.

    Это самый простой вид L-систем. Пример такой системы представлен на рисунке \ref{img:f1} (при построении использовались следующие параметры: аксиома $F$, правило $F: F[-F]F[+F][F]$, количество итераций 3, угол поворота 30).

    \newpage
    \img{85mm}{f1}{Пример простой L-системы}

    \item Параметрические L-системы.

    Каждому символу добавляем переменную (возможно не одну), которая позволяет указывать некоторые параметры для этого символа (возможно величину угла поворота, длину шага, толщину линии и т.п.) Пример такой системы представлен на рисунке \ref{img:f3} (параметры системы: аксиома $BA(0, 0)$, правила $A(x, y) x < y: BA(x + 1, y)$ и $A(x, y) x >= y: +BA(0, y + 1)$, количество итераций 118, угол поворота 90).

    \img{70mm}{f3}{Пример параметрической L-системы}

    \item Контекстно-зависимые L-системы.

    Такие системы принимают во внимание окружение заменяемого символа.

    \item Стохастические L-системы.

    Представляют собой простые системы с возможностью задания вероятности выполнения того или иного правила. Подобный подход вносит некоторый элемент случайности в получающиеся структуры.
\end{enumerate}


\subsection{Алгоритм визуализации L-системы}

Для визуализации предложения используется система рендеринга,
называемая “черепашьей графикой” \cite{hanan}.

Черепашья графика характеризуется размещением в декартовой системе
“черепашки” и передачей ей инструкций. Черепашка движется в соответствии с полученными ею инструкциями и оставляет за собой след. В нашем случае черепашке отправляется каждый символ из предложения L-системы. Ключ к черепашьему языку представлен в таблице \ref{tbl:best}.

\begin{table}[h]
    \begin{center}
        \captionsetup{justification=raggedright,singlelinecheck=off}
        \caption{\label{tbl:best}Алфавит для отрисовки трехмерной системы Линденмайера}
            \begin{tabular}{|l|l|}
            \hline
    {[}A..Z{]}         & \begin{tabular}[c]{@{}l@{}}Любая (не являющаяся константой буква алфавита перемещает\\ черепашку вперед на фиксированное расстояние, перемещаясь\\ черепашка рисует линию)\end{tabular} \\ \hline
    {[}a..z{]}         & \begin{tabular}[c]{@{}l@{}}Переместить черепашку на фиксированное число шагов, не рисуя\\ линию\end{tabular}                                                                            \\ \hline
    +                  & Поворот черепашки вправо на фиксированный угол                                                                                                                                          \\ \hline
    -                  & Поворот черепашки влево на фиксированный угол                                                                                                                                           \\ \hline
    /                  & Крен черепашки вправо на фиксированный угол                                                                                                                                             \\ \hline
    \textbackslash{}   & Крен черепашки влево на фиксированный угол                                                                                                                                              \\ \hline
    \textasciicircum{} & Тангаж черепашки вверх на фиксированный угол                                                                                                                                            \\ \hline
    \_                 & Тангаж черепашки вниз на фиксированный угол                                                                                                                                             \\ \hline
    {[}                & Запись текущего состояния черепашки в стек                                                                                                                                              \\ \hline
    {]}                & \begin{tabular}[c]{@{}l@{}}Извлечение состояния черепашки из стека и присвоение этого\\ состояния\end{tabular}                                                                          \\ \hline
    $\sim$             & \begin{tabular}[c]{@{}l@{}}Нарисовать поверхность, идентифицированную модулем сразу\\ после $\sim$в текущем местоположении и ориентации черепахи.\end{tabular}                          \\ \hline
    \{                 & \begin{tabular}[c]{@{}l@{}}Создать пустой полигон и положить его в стек полигонов (для\\ рисования поверхностей)\end{tabular}                                                           \\ \hline
    G                  & \begin{tabular}[c]{@{}l@{}}Переместить черепашку вперед на фиксированное расстояние и\\ провести линию, не добавляя вершину к текущему многоугольнику.\end{tabular}                     \\ \hline
    g                  & \begin{tabular}[c]{@{}l@{}}Переместить черепашку вперед, не проводя линию и не добавляя\\ вершину к текущему многоугольнику.\end{tabular}                                               \\ \hline
    .                  & Добавить положение черепахи к текущему многоугольнику                                                                                                                                   \\ \hline
    \}                 & \begin{tabular}[c]{@{}l@{}}Извлечь текущий полигон из стека полигонов и нарисовать его,\\ используя указанные вершины. \end{tabular}   \\      \hline                                             
    \end{tabular}
    \end{center}
\end{table}

\newpage

\subsection{Итоговый алгоритм генерации}

Таким образом в итоговом алгоритме генерации модели растения можно выделить следующие шаги:
\begin{enumerate}
    \item Получить параметры растения, представленного в виде L-системы (аксиому, набор продукционных правил, количество итераций роста, начальные данные, длину шага и т.д.)
    \item Используя количество итераций, аксиому и правила получить
предложение, по которому будет строиться итоговое растение.
    \item Если в предложении есть вхождения некоторых поверхностей, убедится, что их представления заданы в программе или сгенерировать их.
    \item “Прочитать” предложение с помощью черепашки и таблицырасшифровки черепашьего языка.
    \item Получить итоговую модель растения, которую уже можно будет
выводить на экран.
\end{enumerate}

\section*{Вывод}
В данном разделе были подробно рассмотрены алгоритмы, которые будут реализованы, и приведены схемы алгоритмов Z-буфера и простой модели освещения, указан способ генерации модели растения.
