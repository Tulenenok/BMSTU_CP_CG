\chapter{Технологический раздел}
В этом разделе будет обоснован выбор языка программирования и среды разработки, рассмотрена диаграмма основных классов и разобран интерфейс, предлагаемый пользователю.

\section{Требования к вводу} 
На вход программы подается набор параметров, описывающих L-систему, визуальной интерпретацией которой является модель растения. В этот набор параметров входят стиль отрисовки, цвет, аксиома, угол поворота на каждой итерации, количество итераций, правила преобразования аксиомы.

\section{Требования к программе} 
Программа должна иметь несколько уже готовых наборов "исходных данных". При этом программа должна предоставлять следующие возможности:
\begin{itemize}
	\item визуальное отображение сцены;
	\item изменение параметров L-системы, описывающей растение;
        \item добавление источника света;
	\item поворот, масштабирование, перемещение исходной сцены;
        \item очистка сцены.
\end{itemize}

Программа не должна аварийно завершаться при некорректном вводе.
Вывод программы - визуальное отображение модели растения, заданной пользователем, на сцене.

\section{Выбор языка программирования и среды разработки}
Для реализации данной программы был выбран язык программирования C++ \cite{Cpp}. На этот выбор повлияли следующие факторы:

\begin{enumerate}
	\item Этот язык предоставляет программисту широкие возможности реализации самых разнообразных алгоритмов. 
        \item C++ обладает высокой эффективностью.
	\item Язык является строго типизированным, что позволяет защититься от неконтролируемых  ошибок.
	\item В данном языке имеется большое количество библиотек и шаблонов, позволяющих не тратить время на изобретение готовых конструкций.
\end{enumerate}

В качестве среды разработки был выбран Clion 2022 \cite{about_clion}. Некоторые факторы по которым была выбрана данная среда:
\begin{enumerate}
	\item Включает весь основной функционал: параллельная сборка, отладчик, поддержка точек останова, сборки и т.д.
	\item Имеет удобный редактор кода со всеми полезными функциями: подсветкой синтаксиса, автоматическим форматированием, дополнением и отступами.
	\item Имеет удобный интерфейс для работы с git.
        \item Данная среда разработки бесплатна для студентов.
\end{enumerate}

\newpage

\section{Структура классов программы}
В программе можно выделить следующие классы:

\begin{itemize}
    \item Object – абстракция базового объекта;
    \item Vertex – абстракция вершины;
    \item Triangle – формализация связи трех вершин;
    \item Plant – класс, реализующий работу с моделью растения;
    \item PlantGenerator – класс, отвечающий за генерацию модели растения;
    \item LightSource – абстракция источника света;
    % \item Camera – абстракция камеры;
    \item Scene - класс для работы со сценой;
    \item InputForm – класс для взаимодействия с пользователем;
    \item MainWindow – основной класс для работы с приложением;
\end{itemize}
  

\img{100mm}{classes}{Диаграмма классов}
\clearpage

\section{Интерфейс}
Интерфейс предоставляет пользователю такие возможности, как добавление модели растения, изменение параметров модели, добавление источника света, поворот сцены, смещение сцены, масштабирование сцены.
На рисунке \ref{img:inter} представлен интерфейс программы.

\img{100mm}{inter}{Интерфейс программы}

\section{Результаты работы программного обеспечения}

На рисунке \ref{img:fern} приведен результат генерации модели папоротника.

\img{110mm}{fern}{Модель папоротника}

На рисунке \ref{img:vodorosl} приведен результат генерации модели синей водоросли.

\img{110mm}{vodorosl}{Модель синей водоросли}

На рисунке \ref{img:verbena} приведен результат генерации модели вербены.

\img{110mm}{verbena}{Модель вербены}


На рисунке \ref{img:spiral} приведен результат генерации модели растениеподобной структуры.

\img{110mm}{spiral}{Модель растениеподобной структуры}

\newpage

\section*{Вывод}
В этом разделе был выбран язык программирования и среда разработки, рассмотрена диаграмма основных классов, разобран интерфейс приложения и приведены результаты работы програмы.