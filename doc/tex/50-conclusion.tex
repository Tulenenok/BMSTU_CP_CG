\chapter*{Заключение}
\addcontentsline{toc}{chapter}{Заключение}

В рамках данного курсового проекта были:% проанализированы и рассмотрены существующие алгоритмы трассировки лучей, а также предложен и реализован способ его улучшения. Выбор варианта улучшения был сделан на основе особенностей сцены данного программного обеспечения.

\begin{itemize}
	\item описаны структуры трехмерной сцены, включая объекты, из которых состоит сцена;
	\item проанализированы и выбраны необходимые существующие алгоритмы для построения сцены;
	\item реализованы выбранные алгоритмы;
	\item разработано программное обеспечение, которое позволит отобразить трехмерную сцену и визуализировать модель растения;
	\item проведены сравнение однопоточного и многопоточного оптимизированного алгоритма удаления невидимых линий.
\end{itemize}

В ходе выполнения эксперимента было установлено, что многопоточная
реализация алгоритма удаления невидимых граней, использующего Z-буфер,
может показывать меньшее время выполнения, чем однопоточная или
линейная реализация. Было выявлено, что выгоднее всего по
времени использовать столько потоков, сколько у процессора логических
ядер.

